\documentclass[11pt]{report}
\usepackage[utf8]{inputenc}
\usepackage{fancyhdr}
\usepackage{color}
\usepackage{listings}
\usepackage{amsmath}
\usepackage{geometry}
\geometry{
 a4paper,
 right=14mm,
 bottom=20mm,
 top=18mm,
}

\lstset{
    frame=single,
    breaklines=true,
    postbreak=\raisebox{0pt}[0pt][0pt]{\ensuremath{\hookrightarrow\space}},
    basicstyle=\ttfamily,
    showstringspaces=false,
}
 
\renewcommand{\chaptername}{Program}

\title{Lab Report \\
      \textbf{Fundamentals Of C Programming} \\
      CSC-102}
\author{\textbf{Anukul Adhikari} \\ 09/22/076}
\date{\today}

\begin{document}
\maketitle

\chapter*{Information}
All code used in this report including source code of this report itself is available at:\\ \texttt{https://github.com/beinganukul/Assignments-Of-C/tree/master/A02}
\section*{Software}
Following compiler and configuration is verified to work with the snippets in this report:\\
\texttt{Compiler - gcc 9.2.0 (GCC) 
\\
Compiler target - x86\_64-pc-linux-gnu
}

\tableofcontents

\chapter{While Loop}
\section{Problem Statement}
Write a program to reverse a number using while loop.
\section{Program}
\lstinputlisting[language=c]{A02Q01.c}


\chapter{Looping The Sentence For n Numbers Of Times}
\section{Problem Statement}
Write a pogram to read an integer number n from keyboard and display the message Get Well Soon n times.

\section{Program}
\lstinputlisting[language=c]{A02Q02.c}


\chapter{Factorial}
\section{Problem Statement}
Write a program to compute the following using factorial.
\begin{enumerate}
  \item factorial of an integer $n$.
\end{enumerate}
\section{Factorial's Algorithm}

The factorial of a positive integer $n$ can be obtained recursively using the following algorithm.
\begin{itemize}
\item If $n=0$, return 1.
\item Multiply $n$ by $(n-1)!$ and return the result. 
\end{itemize}
\section{Factorial}
        \lstinputlisting[language=c]{A02Q03.c}
  
\chapter{Sum Of Natural Numbers}
\section{Problem Statement}
Write a program that asks an integer number n and calculate sum of all natural numbers from 1 to n.

\section{Program}
\lstinputlisting[language=c]{A02Q04.c}


\chapter{Sum of Series using Loop}
\section{Problem Statement}
Compute $1^2+2^2+3^3+...+n^2$ using for loop taking n as input from user.


\section{Program}
\lstinputlisting[language=c]{A02Q04.c}
  
\chapter{Prime Numbers}
\section{Problem Statement}
Write an program to find the prime numbers from given input.
\section{Algorithm}
A natural number is called a prime number (or a prime) if it has exactly two positive divisors, 1 and the number itself. The prime numbers can be computed using following algorithm:
\begin{itemize}
\item Inside each iteration, divide the number by every number less than itself except 1.
\item If remainder $\neq$ 0, the number prime. Otherwise \texttt{break} from the loop.
\end{itemize}
\section{Program}
\lstinputlisting[language=c]{A02Q04.c}

\end{document}
